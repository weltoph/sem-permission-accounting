\section{Permission Accounting}
	%outline:
%	\begin{itemize}
%		%programming related:
%		\item read/write difference
%		\item mutual exclusion
%		%logic/proof related:
%		\item separation logic: conditional critical regions: CCR
%		\item fractional permissions
%		\item counting permissions
%		\item combine both approaches
%	\end{itemize}

	\emph{Separation Logic} as introduced above is used to reason about mutable data
	structures in the \emph{Heap} which is therefore separated into independent
	parts. This separation can be used to address concurrency by separating
	the \emph{Heap} and executing every concurrent program on its "own" part
	of the \emph{Heap}. Furthermore a system of permissions is introduced to
	share memory between concurrent threads as well as mechanisms to provide
	mutual-exclusive access on memory. 

	\subsection{Concurrency in \emph{Separation Logic}}
	This approach on concurrency in \emph{Separation Logic} is predominantly
	based on \cite{ConcurrencyInSepLogic} and \cite{PermAcc}.

	Concurrently executing programs which access the same memory, threaten to
	be \emph{racy}. This means that the values of variables or memory cells might not
	be only dependent on the commands of the programs but also from unintended
	access to shared memory by other programs. This may occur if two or more
	programs access shared memory at the same time, then the result of both
	operations depend on surrounding factors like for example the scheduler of
	the machine. \emph{Racy} programs are not easily reasoned about because it requires
	deep knowledge about the current state of the executing machine which might
	differ for every execution. Therefore it seems reasonable to have a guaranty that
	concurrent programs are not \emph{racy}, which is called \emph{race-free}.

	In order to access variables \emph{race-freely} from concurrent programs
	\cite{PermAcc} states that the variables
	need to be part of one of the following three groups of variables:
	Firstly, variables which are read and written exclusivly by one
	program. Secondly, variables which are accessed from concurrent programs but
	are only read. And thirdly, variables which are accessed from concurrent
	programs but where the write access is only allowed in a mutually-exclusive
	section of the program, i.e. a section which guarantees that in the
	meantime no other thread accesses those variables.

	In what follows a syntax for concurrent execution of programs is introduced as:
	\begin{mydef}[Concurrent Execution]
		For two programs $C_1,C_2$
		$$C_1\parallel C_2$$
		defines \emph{concurrent execution} of $C_1$ and $C_2$.
	\end{mydef}

	As stated above programs with variables read and written exclusivly by
	the program itself are race-free, also multiple programs which only
	\underline{read} from the same variables are race-free. This means
	race-freedom can be guaranteed for multiple concurrently executed programs
	if they only read from shared memory and if every modified variable is only
	accessed exclusively by one of the programs.

	\begin{mydef}[Concurrency Rule]
	Given $n$ programs $C_1,\dots,C_n$ with individual
	\emph{preconditions} $Q_1,\dots,Q_n$ and corresponding \emph{postconditions} $R_1,\dots,R_n$ where for
	every $i$: $\{Q_i\}C_i\{R_i\}$ is partially correct and race-free. If
	$\forall i,j \in \{1..n\}, i \neq j.(\text{free}(Q_i)\cup\text{free}(R_i))
	\cap \text{modifies}(C_j)  = \emptyset$ then
	$\{Q_1\ast\cdots\ast Q_n\}(C_1\parallel \cdots\parallel C_n)\{R_1\ast\cdots\ast R_n\}$
	is also partially correct and race-free. Stated as inference rule:
	\begin{prooftree}
	\AxiomC{$\{Q_1\}C_1\{R_1\}\cdots\{Q_n\}C_n\{R_n\}$}
	\UnaryInfC{$\{Q_1\ast\cdots\ast Q_n\}(C_1\parallel \cdots\parallel
		C_n)\{R_1\ast\cdots\ast R_n\}$}
	\end{prooftree}
	\end{mydef}
	\begin{myexp}[Concurrently executed Assignmend]
	Just to convay a basic intuition on using the concurrency rule
	the following example shows how three memory cells are assigned
	different values concurrently
	\begin{lstlisting}[mathescape]
x := new();
y := new();
z := new();
[x] := 5; $\parallel$ [y] := 7; $\parallel$ [z] := 13;
	\end{lstlisting}
	The program is obviously rather simple but it may help to examine an
	easy example first. Now assertions are added, and as well some comments
	to explain details:
	\begin{lstlisting}[mathescape, escapeinside={-<}{>-}]
-<\textcolor{green}{$\{\textit{emp}\}$}>-
x := new();  // Axioms of Referenced Access
-<\textcolor{green}{$\{x\mapsto -\}$}>-
-<\textcolor{green}{$\therefore\{x\mapsto - \ast \textit{emp}\}$}>-
// emp is the neutral element of the operator $\ast$ and can 
// therefore be added to any heap
y := new(); // Axioms of Referenced Access
-<\textcolor{green}{$\{x\mapsto - \ast y \mapsto -\}$}>-
// Note that new() is expected to return another value for $y$ than for $x$
// because the cell $x$ points to is already allocated, and therefore $x\mapsto -$
// and $y\mapsto -$ can be separated with $\ast$
-<\textcolor{green}{$\therefore\{x\mapsto - \ast y\mapsto - \ast \textit{emp}\}$}>-
z := new(); // Axioms of Referenced Access
-<\textcolor{green}{$\{x\mapsto - \ast y\mapsto - \ast z\mapsto -\}$}>-
[x] := 5; $\parallel$ [y] := 7; $\parallel$ [z] := 13;
-<\textcolor{green}{$\{x\mapsto 5 \ast y\mapsto 7 \ast z\mapsto 13\}$}>- // Concurrency Rule
	\end{lstlisting}
	Actually, the side conditions of the \emph{Concurrency Rule} needs to be
	checked to apply it. But this is omitted due to its simplicity.
	\end{myexp}

	But as stated above programs are also race-free if the access to a variable
	or shared memory is guaranteed to be mutual-exclusive. In order to provide
	such a mutual-exclusive access
	the formal language which is used in this paper to define programs, is
	extended with a mechanism of \emph{Conditional Critical Region(CCR)}. A
	\emph{CCR} is used to provide mutual-exclusive access to resources, namely
	variables and shared memory.
	Those resources can be collected in a set which is called a \emph{bundle}.
	For a \emph{bundle} $b$, a boolean expression $G$ and a program $C$ a
	\emph{CCR} is a command of the form:
	\begin{lstlisting}[mathescape]
with $b$
when $G$ do
	$C$
od
	\end{lstlisting}
	The idea is to provide mutual-exclusive access to all resources in the
	\emph{bundle} $b$ within $G$ and $C$. This means that the 
	resources in $b$ must not be accessed from outside the \emph{CCR},
	but can be accessed from inside with the guarantee that there is no
	other access at the same time from another program.

	A \emph{CCR} is handled in the following way:
	\begin{enumerate}
		\item the \emph{bundle} $b$ is aquired, i. e. an exclusive access
			on all items of the \emph{bundle} is guaranteed
		\item the condition $G$ is evaluated
		\item if $G$ is true, the program $C$ is executed and afterwards $b$ is
			released
		\item if $G$ is false, $b$ is released and the execution continues at step 1
	\end{enumerate}
	Having this in mind it is rather intuitive that this allows race-free sharing
	of resources, because of step 1. But to link this syntactical element to
	logical assertions, it is helpful to give another reading to the
	"$\mapsto$ operator": It can be understood as the permission to access the
	memory cell.
	And this right to access the memory cells of $b$ is only granted within the
	\emph{CCR}. Therefore to prove properties of a \emph{CCR} one needs an
	invariant $I_b$ for a \emph{bundle} $b$ which is only accessible within the
	\emph{CCR} and cannot be accessed from outside the \emph{CCR}. This invariant
	is used in a similar way as in the proofs for loops in \emph{Hoare Calculus},
	because if an invariant exists which all concurrent programs respect, this
	assertion can always be guaranteed. The following rule formalises that the
	variables of a \emph{bundle} can only be accessed within a \emph{CCR}:
	
	\begin{mydef}[CCR Rule]
		For a \emph{bundle} $b$, a boolean expression $G$, a program $C$,
		assertions $Q, R$ and an invariant $I_b$, the following inference rule
		holds regarding race-freedom and partial correctness:
	\begin{prooftree}
		\AxiomC{$\{(Q\ast I_b) \land G\} C \{R \ast I_b\}$}
		\UnaryInfC{$\{Q\}\text{with $b$ when $G$ do $C$ od}\{R\}$}
	\end{prooftree}
	\end{mydef}

\subsection{Abstract Permission Model}
	The abstraction of \emph{Permission Models} is mainly based on
	\cite{freshlook} and \cite{PermAcc}.

	As already mentioned the "$\mapsto$ operator" can be read as a permission to
	access a certain memory cell. In the following this reading is expanded to
	distinguish between read and write access to a cell. By now a memory cell can always
	be part of only one independent part of the \emph{Heap} which is separated
	from others by $\ast$. But a cell which is only read can be logically
	associated with multiple part of the \emph{Heap} where all parts of the 
	\emph{Heaps} need to agree on the value of the cell. This can be used to
	formulate assertions for programs which only need a read access on a
	cell. Obviously multiple possibilities to implement such a
	\emph{Permission Accounting} exist. But all of these implementations need to
	have certain properties. These properties are now defined and later on
	different possibilities of actually realising a permission model are
	introduced.

	A model which allows \emph{Permission Accounting} needs to manage permission
	associated with every access on cells of the \emph{Heap}. Therefore the
	definition of \emph{Heaps} slightly adapts to
	\begin{mydef}[Heap]
		For a set of addresses $L$, a set of values $V$ and a set of
		permissions $M$ a \emph{Heap} is defined as partial function
		$$h: L \rightharpoonup (V\times M)$$
	\end{mydef}
	\noindent
	Note that the set of addresses and the set of values are not explicitly
	defined. For all \emph{Heaps} above always applied $L = \mathbb{N}$ and
	$V = \textit{Integers}$. But although addressable memory cells are very
	intuitive, the dynamically allocatable memory of programs might be
	represented differently. For example with \emph{Petri Nets} \cite{seplogproof}.

	Regarding the set of permissions $M$ the intuitive idea is that
	exactly one \emph{source permission} exists which grants full access rights
	on a cell including read and write permission as well as the permission to
	dispose the cell. From a \emph{source permission} multiple read permissions
	can be derived. Deriving a read permission from a \emph{source permission}
	causes the \emph{source permission} to loose the right of writing and
	disposing the cell, it becomes a read permission as well to avoid \emph{racy}
	behaviour. A \emph{source permission} can always be regained by recombining
	all permissions of a cell.
	This is mathematically represented by a binary operator for the elements of
	$M$ which adds permissions together and can be used to
	define splitting of a permission. This operator is commutative, because the
	order of how permissions are recombined does not matter. There is no neutral
	element to this operator because the not granted permission is always
	implicitly stated by not mentioning a cell in an assertion.
	Additionally, certain elements can not be combined, for example two
	\emph{source permissions} for the same cell. Mathematically speaking:
	\begin{mydef}[Permission]
		A set of permissions $M$ is equipped with a partial commutative
		semigroup structure with the binary operator $\oplus$.
	\end{mydef}
	The \emph{source permission} needs to be unique and can not be expanded
	further because it already is the maximal permission. Futhermore every
	permission can be recombined to a \emph{source permission} again:
	\begin{mydef}[Source permission]
		In a set of permissions $M$ exists exactly one $m_W$ with:
		$$\forall m\in M.m_W\oplus m\text{ is undefined}$$
		$$\forall m\in M\setminus \{m_W\} \exists m'. m \oplus m' = m_W$$
	\end{mydef}
	This permission model is now connected to cells by expanding the operator
	$\oplus$ to the set $(V\times M)$ as
	$$(v,m)\oplus(v',m') =
		\begin{cases}
			(v,m\oplus m') &\text{ if $v=v'$ and $m\oplus m'$ \textit{defined}}\\
			\text{undefined} &\text{ otherwise}\\
		\end{cases}
	$$
	Note that two cells can only be combined if they agree on the value $v$.

	Correspondingly, the operator is also applied to \emph{Heaps} but to fit
	the definitions above it is named $\ast$:
	\begin{mydef}[Separation of Heaps]
		If for two \emph{Heaps} $h,h'$ for each $l\in \textit{dom}(h)\cap\textit{dom}(h')$ $h(l)\oplus h'(l)$
		is defined the \emph{Separation} of those(written as $h\ast h'$)
		is defined as:
		$$(h\ast h')(l) =
			\begin{cases}
				h(l) &\text{ if $h'(l)$ undefined}\\
				h'(l)&\text{ if $h(l)$ undefined}\\
				h(l) \oplus h'(l) &\text{ otherwise}\\
			\end{cases}
		$$
	\end{mydef}

	This abstraction of \emph{Permission Accounting} needs to be connected to
	the programming syntax. For convenience the following syntax will be used
	for a program $C$, \emph{Stores} $s,s'$ and \emph{Heaps} $h,h'$:
	$C,s,h \leadsto s', h'$. This means the command $C$ executed on a
	\emph{Heap} $h$ and a \emph{Store} $s$ leads to a \emph{Heap} $h'$ and a
	\emph{Store} $s'$. Using this new syntax the commands of the programming
	language can be defined as follows:
	\begin{mydef}[Semantics of commands]
	For a \emph{Store} $s$, a set of values $V$, a set of addresses $L$, a
	set of permissions $M$ with a maximum permission of $m_W$, a \emph{Heap}
	$h:L \rightharpoonup (V\times M)$, integer expressions $E,E'$ and $v\in V$,
	$l\in L$,$x\in \textit{Vars}$ the definition of the commands are:\\
		\begin{center}
		\begin{prooftree}
		\AxiomC{$\llbracket E \rrbracket s = v$}
		\UnaryInfC{$x:=E,\; s,h \leadsto (s|x\mapsto v),h$}
		\end{prooftree}
		\end{center}
		\begin{center}
		\begin{prooftree}
		\AxiomC{$\llbracket E'\rrbracket s = l$}
		\AxiomC{$\llbracket E \rrbracket s = v$}
		\AxiomC{$h(l) = (-,m_W)$}
		\TrinaryInfC{$[E'] := E,\; s,h \leadsto s,(h|l\mapsto(v,m_W))$}
		\end{prooftree}
		\end{center}
		\begin{center}
		\begin{prooftree}
		\AxiomC{$\llbracket E'\rrbracket s = l$}
		\AxiomC{$h(l) = (v, m)$}
		\BinaryInfC{$x:=[E'],\; s,h \leadsto (s|x\mapsto v),h$}
		\end{prooftree}
		\end{center}
		\begin{center}
		\begin{prooftree}
		\AxiomC{$l \in L \setminus \text{dom}(h)$}
		\AxiomC{$\llbracket E \rrbracket s = v$}
		\BinaryInfC{$x:=\text{new}(E),\; s,h \leadsto (s|x\mapsto l),(h|l\mapsto(v,m_W))$}
		\end{prooftree}
		\end{center}
		\begin{center}
		\begin{prooftree}
		\AxiomC{$\llbracket E'\rrbracket s = l$}
		\AxiomC{$h(l) = (v,m_W)$}
		\BinaryInfC{$\text{dispose } E',\; s,h \leadsto s, (h\setminus{\{(l,v)\}})$}
		\end{prooftree}
		\end{center}
	\end{mydef}
	\noindent
	Note that these are the definitions of the commands as above with enforced
	permissions. A \emph{source permission} is necessary for changing the content
	of and disposing a cell. Allocating a memory cell initially grants a
	\emph{source permission}. Also this semantics are restricted to the
	sequential execution of the programs.

	These semantics of commands yield the following axiomatic Hoare-triples,
	where the "$\mapsto$ operator" is decorated with a permission $z$ as
	$\xmapsto{z}$:
	\begin{mydef}[Axioms for Regulated Heap Access]
	For an assertion $P$, a variable $x$, value expressions $E, E'$, which are
	expressions that can be evaluated to a value from the value-set $V$, values
	$v, n$, a permission $m$, a \emph{source permission} $m_W$ and a
	\emph{Store} $s$ the axioms for a permission model $M$ are:\\
	\begin{center}
	\begin{tabular}{p{4cm}p{3cm}p{5cm}}
		$\{P[E/x]\}$ & $x:=E$ & $\{P\}$ \\
		$\{E\xmapsto{m_W}-\}$ & $[E'] := E$ & $\{E'\xmapsto{m_W}E\}$ \\
		$\{E\xmapsto{m} n \land \llbracket x\rrbracket s =v\}$ & $x:=[E]$ & $\{E[v/x]\xmapsto{m}n\land\llbracket x\rrbracket s = n\}$\\
		$\{\textit{emp}\}$ & $x:=\textit{new}()$ & $\{x \xmapsto{m_W} -\}$\\
		$\{E\xmapsto{m_W}\}$ & $\textit{dispose}(E)$ & $\{\textit{emp}\}$\\
	\end{tabular}
	\end{center}
	\end{mydef}

	\subsection{Permission Models}
	The different \emph{Permissions Models} are defined in \cite{freshlook} and
	\cite{PermAcc}.

	An example for a model of \emph{Permission Accounting} is the model of
	\emph{Fractional Permissions}. The intuitive idea of \emph{Fractional Permissions}
	is that $1$ is the \emph{source permission} and every permission can be split
	into fractions to grant read permissions. Every permission
	$z \in (0,1)\cap\mathbb{Q}$ is a read permission which can be split further.
	Now, $\oplus_\text{FP}$ needs to be defined but this is rather
	straightforward as:
	\begin{mydef}[Combining Fractional Permissions]
		For two permissions $z,z' \in (0,1]\cap\mathbb{Q}$ the combination
		$z \oplus_\text{FP} z'$ is defined as:
		$$z \oplus_\text{FP} z' =
			\begin{cases}
				\text{undefined} & \text{ if $z + z' > 1$} \\
				z + z' & \text{ otherwise} \\
			\end{cases}
		$$
	\end{mydef}
	The formal model for $(M, m_W, \oplus)$ is
	$((0,1]\cap\mathbb{Q},1,\oplus_\text{FP})$.

	Another model for \emph{Permission Accounting} is the model of
	\emph{Counting Permissions}. Here the \emph{source permission} gives away
	tokens of read permissions and counts how many of these tokens are given
	away. The \emph{source permission} is $0$. Read permissions are indicated
	by negative numbers and with every derived read permission the token counter
	is incremented for the \emph{source permission} which just means that the
	permission is incremented by $1$. This leads to a special role of the
	\emph{source permission} because it is the only permission indicated by
	a positive number(or $0$). It can be understood as a central instance which
	gives away read accesses, where for \emph{Fractional Permissions} are a
	decentralised approach because every split produces two homogen
	tokens. Formally \emph{Counting Permissions} are
	$(\mathbb{Z},0,\oplus_\text{CP})$ with
	$$z \oplus_\text{CP} z' =
		\begin{cases}
			\text{undefined} & \text{ if $z \geq 0$ and $z' \geq 0$}\\
			\text{undefined} & \text{ if ($z\geq 0$ or  $z' \geq 0$) and $z + z' < 0$}\\
			z + z'           & \text{ otherwise}
		\end{cases}
	$$

	In \cite{freshlook} other models for \emph{Permission Accounting} are
	mentioned like the \emph{Boolean Share Model} which models how
	\emph{Separation Logic} was introduced above without
	\emph{Permission Accounting}.

	\emph{Permission Accounting} is used to restrict access to memory cells.
	A restricted access to a memory cell, for example a read access can be
	given to a program. And every command of the program must not change
	the value of the memory cell. The property that a command does not change
	the value of a memory cell is called \emph{Passivity}:
	\begin{myprop}[Passivity]
		A command $C$ shows \emph{Passivity} on a cell if it has access to
		a cell but does not change it.
	\end{myprop}
	\begin{mylem}[Passivity of cells with read permissions]
		If a command $C$ has only read access to a memory cell, it shows
		\emph{Passivity} on that cell.
	\end{mylem}
	\begin{myproof}[Passivity of cells with read permissions]
		Let $C$ be a program, $l$ an address, $v,v'$ values and $z,z'$
		permissions with $z\neq m_W$, $z'\neq m_W$, $z \oplus z' = m_W$,
		$v \neq v'$. And for $C$ two possibilities can be distinguished:
		Firstly, $C$ is a sequential program. In that case \emph{Termination
		Monoticity} (formally introduced in \ref{sec:soundness}) is given,
		which means if $C$ terminates in a \emph{Heap} $h$ it also terminates
		in a \emph{Heap} $h\ast h'$ for $h'$ is a compatible \emph{Heap}.
		Assume
		$$\{l\xmapsto{z}v\}C\{l\xmapsto{z}v'\}$$
		is a partially correct and terminating Hoare-triple. With the
		\emph{Frame Rule} follows
		$$\{l\xmapsto{z}v \ast l\xmapsto{z'}v\}C\{l\xmapsto{z}v' \ast l\xmapsto{z'} v\}$$
		Where \emph{Termination Monoticity} states that this Hoare-triple terminates
		and the \emph{Frame Rule} states that this Hoare-triple is partially correct.
		But the \emph{postcondition} is obviously false, and therefore there is a
		contradiction, which means that $C$ can not change the value of the
		memory cell with address $l$. The memory cell shows \emph{Passivity}.

		Secondly, $C$ is concurrently executed program, which means $C$ might aquire
		further access permissions by using permissions from a \emph{bundle}
		$b$. This case is dealt with exemplatory, but the proof can be
		accordingly generalised:
		For the \emph{bundle} $b$ an invariant $I_b \equiv l\xmapsto{z'}-$ exists
		and $C \equiv \text{with }b\text{ when true do } [l] := v' \text{ od}$.
		Using the \emph{CCR Rule} one can show that:
		$$\{l\xmapsto{z}v\}C\{l\xmapsto{z}v'\}$$
		by the following proof:

		\begin{lstlisting}[mathescape]
$\{l\xmapsto{z}v\}$
with $b$ when true do
	$\{(l\xmapsto{z}v\ast l\xmapsto{z'}-) \land true\}$
	$\therefore\{l\xmapsto{m_W}v\}$
	$[l] := v';$
	$\{l\xmapsto{m_W}v'\}$
	$\therefore\{l\xmapsto{z}v'\ast l\xmapsto{z'}-\}$
od
$\{l\xmapsto{z}v'\}$
		\end{lstlisting}

		With the \emph{Frame Rule} it is possible to prove the  Hoare-triple:
		$$\{l\xmapsto{z}v\ast l\xmapsto{z'}v\}C\{l\xmapsto{z}v'\ast l\xmapsto{z'}v\}$$
		Now if $C$ is part of a set of concurrently executed programs as:
		$C_1,\dots,C_n$ with $\exists j\in \{1,\dots,n\}.C_j = C$. Let $Q_1,\dots,Q_n$
		be the according \emph{preconditions} and $R_1,\dots,R_n$ the \emph{postconditions}.
		The \emph{Concurrency Rule} would lead for $(C_1\parallel\cdots\parallel C_n)$
		to a \emph{precondition} with
		$\{\cdots\ast I_b \ast l\xmapsto{z}v \ast l\xmapsto{z'}v \ast\cdots\}$. This
		\emph{precondition} is not satisfyable because such a \emph{Heap} does not
		exist. Therefore $C$ can not be part of a concurrent execution but this
		was assumed, which means for a read permission \emph{Passivity} on the cell
		is guaranteed.
	\end{myproof}

	\subsection{Soundness and race-freedom}
	\label{sec:soundness}
	After having introduced \emph{Separation Logic} and
	\emph{Permission Accounting} as logical tools to reason about programs which
	can be executed concurrently, share memory between each other and access
	dynamically allocated memory, a proof for the soundness of the introduced
	rules and the race-freedom of the approach of \emph{Permission Accounting}
	would justify these approaches. This proof is given in \cite{seplogproof}
	and due to its complexity it can not be given here.
	Instead to convay an intuition on how to prove properties for
	\emph{Separation Logic} a few definitions regarding soundness of rules for
	sequentially executed programs which are based on \cite{sembasis} are
	introduced and exemplatory the \emph{Frame Rule} is examined.

	To examine formally the execution of a program it is understood as sequence
	of configurations. Configurations are current state of a machine, which is
	the state of the \emph{Heap} $h$, the state of the \emph{Store} $s$ and
	potentially a sequence of commands (a program) $C$. A configuration $(s, h)$
	where no more commands exist to be executed is called \emph{terminal}. A non
	\emph{terminal} configuration is a triple $(C, s, h)$. On the set of
	configurations the relation $\leadsto$ which was already utilized
	above for defining the semantics of the commands in \emph{Permission
	Accounting} is defined as the change of the configuration based on the
	executed command. In the previous definitions a program $C$ was expected to
	be well defined which means it does not access memory cells which are not
	allocated but now the set of configurations can be extended with a
	\textit{fault} state. The \textit{fault} state is reached if a memory cell
	not present in the \emph{Heap} $h$ is accessed. To get an intuition on
	configurations and $\leadsto$ a few relations are given:
	\begin{mydef}[Configuration Transitions]
	For a \emph{Heap} $h$, a \emph{Store} $s$, an integer expression $E$, a
	boolean expression $B$ and programs $C, C'$ the following relations on
	configurations are defined:
	\begin{enumerate}
	\item \begin{prooftree}
			\AxiomC{$\llbracket E \rrbracket s \notin \textit{dom}(h)$}
			\UnaryInfC{($x := [E],s,h)\leadsto \textit{fault}$}
			\end{prooftree}
	\item \begin{prooftree}
			\AxiomC{$\llbracket B \rrbracket s \equiv \textit{true}$}
			\UnaryInfC{(while $B$ do $C$ od, $s, h)\leadsto$(($C$; while $B$ do $C$ od), $s$, $h)$}
			\end{prooftree}
	\item \begin{prooftree}
			\AxiomC{$\llbracket B \rrbracket s \equiv \textit{false}$}
			\UnaryInfC{(if $B$ then $C$ else $C'$, $s,h)\leadsto (C', s, h)$}
			\end{prooftree}
	\end{enumerate}
	\end{mydef}

	The relation $\textit{Conf} \leadsto^\ast \textit{Conf'}$ states that
	there is a $\leadsto$-sequence of configurations from \textit{Conf} to
	\textit{Conf'}.
	In the following a configuration is called \emph{safe} if the execution from
	this configuration does not lead to a fault state.
	\begin{mydef}[Safety]
		A configuration $(C,s,h)$ is \emph{safe} if $(C,s,h)$ $\slashed{\leadsto}^\ast$ \textit{fault}.
	\end{mydef}
	This can be used to expand partial correctness to grand safety as well:
	\begin{mydef}[Partial Correctness with Safety]
		The Hoare-triple $\{P\}C\{R\}$ for a \emph{precondition} $P$, a
		\emph{postcondition} $R$ and a program $C$ is \emph{partially correct}
		if for all \emph{Stores} $s$ and \emph{Heaps} $h$ applies:
		if $s,\;h\models P$ then
			$(C,s,h)$ is safe, and if $(C,s,h)\leadsto^\ast (s',h')$ for a
			\emph{Store} $s'$ and a \emph{Heap} $h'$ then $s',\;h'\models R$.
	\end{mydef}
	As already mentioned programs exist which do not terminate. Such programs
	have an infinite $\leadsto$-sequence from their starting configuration.
	All programs which terminate for every state and which are partially correct
	are called totally correct:
	\begin{mydef}[Total Correctness]
		The Hoare-triple $\{P\}C\{R\}$ for a \emph{precondition} $P$, a
		\emph{postcondition} $R$ and a program $C$ is \emph{totally correct}
		if for all \emph{Stores} $s$ and \emph{Heaps} $h$ applies:
		if $s,\;h\models P$ then there is no infinite $\leadsto$-sequence
		starting from $(C,s,h)$ and if $(C,s,h) \leadsto (s',h')$ for
		a \emph{Store} $s'$ and a \emph{Heap} $h'$ then $(s',h')\models R$.
	\end{mydef}
	In order to prove the \emph{Frame Rule} sound for partial and total
	correctness it is essential to prove the locality of commands. Locality
	is the property that commands executed on one part of the \emph{Heap}
	do not affect any other part of the \emph{Heap} and that the behaviour of
	the command is not affected by any other part of the \emph{Heap}. This
	regards safety as well as termination of programs. Formally formulated
	as the following properties:
	\begin{myprop}[Safety Monotonicity]
		For a program $C$, a \emph{Store} $s$ and \emph{Heaps} $h,h'$:
		If $(C,s,h)$ is safe and $h\#h'$, then $(C,s,h\ast h')$ is safe.
	\end{myprop}
	\begin{myprop}[Termination Monotonicity]
		For a program $C$, a \emph{Store} $s$ and \emph{Heaps} $h,h'$:
		If there is no infinite sequence of configurations starting from
		$(C,s,h)$ and $h\#h'$, then there is no infinite sequence of
		configurations starting from $(C,s,h\ast h')$.
	\end{myprop}
	\begin{myprop}[Frame Property]
		For a program $C$, \emph{Stores} $s,s'$, \emph{Heaps} $h_0,h_1,h'$:
		If $(C,s,h_0)$ is safe and $(C,s,h_0\ast h_1) \leadsto^\ast (s',h')$
		then a \emph{Heap} $h_0'$ exists with $(C,s,h_0)\leadsto^\ast (s',h_0')$
		and $h' = h_0'\ast h_1$.
	\end{myprop}
	These properties are sufficient to prove the \emph{Frame Rule} sound by the
	following argument:
	\begin{mylem}[Soundness of the \emph{Frame Rule}]
		From \emph{Safety Monotonicity}, \emph{Termination Monotonicity} and
		the \emph{Frame Property} follows the soundness of the \emph{Frame Rule}
		for partial and total correctness.
	\end{mylem}
	\begin{myproof}
		Suppose $\{P\}C\{Q\}$ is a totally correct Hoare-triple. This means
		that for any \emph{Heap} $h_P$ and \emph{Store} $s$ with $(s,h_P)\models P$:
		$(C,s,h_P)\leadsto^\ast (s',h')$ gives us $(s',h')\models Q$. Let $s$
		be a \emph{Store}.
		Now let $h_R$ be any \emph{Heap} with $(s,h_R)\models R$. And $h_P$ any
		\emph{Heap} with $(s,h_P)\models P$. Because $(C,s,h_P)$ is safe it follows
		from \emph{Safety Monoticity} that $(C,s,h_P\ast h_R)$ is safe.
		\emph{Termination Monoticity} gives the existence of a \emph{Heap} $h$ with
		$(C,s,h_P\ast h_R) \leadsto^\ast (s',h)$. And for this \emph{Frame Property}
		yields $h_P'$ with $h_P'\ast h_R = h$ and $(C,s,h_P)\leadsto^\ast (s',h_P')$.
		Because $\{P\}C\{Q\}$ is totally correct and $(C,s,h_P)\leadsto^\ast (s',h_P')$
		it follows that $(s',h_P')\models Q$. Because of the side condition of the
		\emph{Frame Rule} the \emph{Stores} $s, s'$ agree on the values of every
		variable free in $R$. This means because $(s,h_R)\models R$ that $(s',h_R)\models R$.
		This gives $(s',h)\models Q\ast R$. Therefore the \emph{Frame Rule} is sound
		for partial and total correctness.
	\end{myproof}
	This means if the properties of \emph{Termination Monoticity},
	\emph{Safety Monoticity} and \emph{Frame Propery} can be proven for
	\emph{Heaps} build from \emph{Permission Accounting} the soundness of the
	\emph{Frame Rule} is given. As \cite{PermAcc} states the condition
	$$\forall m\in M. m_W\oplus m \text{ is undefined}$$
	ensures \emph{Frame Property} and \emph{Safety Monoticity}. This can be
	shown by the following arguments:
	\begin{myproof}[Frame Property of \emph{Heaps} with \emph{Permission Accounting}]
		Let the premisses of the \emph{Frame Property} hold. In the following
		$h, h_0, h_1, h_0'$ refers to \emph{Heaps} and $s$ to a \emph{Store}.

		For $[E'] := E$: Because $([E'],s,h_0)$ is safe by assumption, for a value $v$:
		$h_0(\llbracket E'\rrbracket s) = (v,m_W)$. This means that for $h_1$
		holds $\llbracket E'\rrbracket s \notin \textit{dom}(h_1)$ and therefore
		$[E'] := E$ does not affect $h_1$. With $h_0' = (h_0|\llbracket E'\rrbracket s = E)$
		gives $([E'] = E,s,h_0)\leadsto^\ast s', h_0'$ and $h' = h_0'\ast h_1$.

		For $\textit{dispose}(E)$: Let $l = \llbracket E \rrbracket s$. Because
		$(\textit{dispose}(E),s,h_0)$ is safe, a value $v$ exists with 
		$h_0(l) = (v, m_W)$. Therefore $l \notin \textit{dom}(h_1)$ and for
		$h_0'$ is the \emph{Heap} for which $h_0 = h_0' \ast l\mapsto v$ follows
		$(\textit{dispose}(E),s,h_0)\leadsto^\ast s', h_0'$ and $h' = h_0'\ast h_1$.

		For $x := [E]$: Let $l = \llbracket E \rrbracket s$ and because of the
		Safety of $(x:=[E],s,h_0)$ follows $l\in\textit{dom}(h_0)$ and with
		$h_0' = h_0$ the \emph{Frame Property} follows immediatly.

		For $\text{\emph{if }}B\text{\emph{ then }} C\text{\emph{ else }} C'$,
		$\text{\emph{while }}B\text{\emph{ do }} C \text{\emph{ od}}$, $C_1;C_2$:
		Consider a variant of \emph{Frame Property} as $(C,s,h_0)$ is
		safe and $(C,s,h_0\ast h_1) \leadsto^\ast (C',s',h')$ then there is a
		\emph{Heap} $h_0'$ with $(C,s,h_0)\leadsto^\ast (C',s',h_0)$ and $h'=h_0'\ast h_1$.
		Now the proof works with induction by the derivation steps for \emph{while}
		and \emph{if} and \emph{;} :
		Let $(\text{\emph{while }} B \text{\emph{ do }} C \text{\emph{ od}},s,h_0)$
		be safe and $(\text{\emph{while }} B \text{\emph{ do }} C \text{\emph{ od}},s,h_0)
		\leadsto^\ast(C',s',h')$. If $B$ is false this is one of the atomic cases
		for $(C, s, h_0)$, thus, $B$ is true:
		$(\text{\emph{while }} B \text{\emph{ do }} C \text{\emph{ od}},s,h_0)\leadsto
		 (C;\text{\emph{while }} B \text{\emph{ do }} C \text{\emph{ od}},s,h_0)$
		 and with the induction hypothesis for this deriveration \emph{Frame Property}
		 or its variant is already shown. This can be executed
		 analogically for \emph{if} and \emph{;}.
	\end{myproof}

	\begin{myproof}[Safety Monoticity of \emph{Heaps} with \emph{Permission Accounting}]
		If the configuration $(C,\;s,\;h)$ is safe there are two possiblities for
		every $l\in\textit{dom}(h)$:
		Firstly, the cell with address $l$ is only read. Then this right can not
		be taken away by adding another \emph{Heap} $h'$. This means $(C,s,h\ast h')$
		is safe as well.
		Secondly, the cell with address $l$ is written, which means that for a
		value $v$: $h(l) = (v, m_W)$ but this implies that for every \emph{Heap}
		$h'$ for which $h\ast h'$ is defined, holds: $l \notin\textit{dom}(h')$.
		Therefore $(C,s,h\ast h')$ is safe.
	\end{myproof}
	The proof of \emph{Termination Monoticity} works analogically to the proof
	of \emph{Safety Monoticity}.
	In conclusion the \emph{Frame Rule} is sound for \emph{Separation Logic} with
	\emph{Permission Accounting} for sequential programs.
	For a complete proof of soundness and race-freedom of concurrent programs
	in \emph{Separation Logic} with \emph{Permission Accounting} see \cite{seplogproof}.

