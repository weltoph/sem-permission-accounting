\section{Introduction}
%	\begin{itemize}
%		\item Concurrency
%		\item shared access variables
%		\item logical approach, reasoning about heaplets separatly $\rightarrow$ Separation Logic
%	\end{itemize}
	Because the requirements for computer programs regarding their performance
	as well as their correctness increase there are two recent developments.
	Firstly, to increase the performance programs rely more and more on
	concurrency. Secondly, the verification of software becomes more and more
	important. To address verification of programs the \emph{Hoare Calculus}
	was developed as possibility to reason about programs. But with increasing
	complexity of programs and the usage of the dynamically managed memory of
	machines the \emph{Hoare Calculus} was not the appropiated tool to verify
	those programs. Additionally, due to the increasing importance of concurrency
	new challenges arise like dealing with race conditions and shared memory.
	To address these challenges \emph{Separation Logic} was introduced which
	extends \emph{Hoare Calculus} with possibilities to reason about mutable
	data in the dynamically managed memory of the machine. This is achieved by
	separating this memory in independent parts which can be examined
	individually. And because these parts are independent concurrent programs
	can work on separated parts of the memory without interfering with each
	other. Therefore \emph{Separaion Logic} can be applied to reason about
	mutable data structures as well as concurrently executed programs. Additionally,
	it is sometimes reasonable to allow multiple concurrent programs to share
	memory. In order to deal well with shared memory a mechanism of \emph{Permission
	Accounting} is used which gives a possibility to limit access to memory.
	This is used to avoid race conditions.
	
	This paper starts with an introduction
	on the \emph{Hoare Calculus} and the extention, \emph{Separation Logic}.
	Further on, it is dealt with concurrency in \emph{Separation Logic} by
	defining the appropiated mechanisms to provide the logical basis to reason
	about shared memory. This leads to \emph{Permission Accounting} which gives
	a possibility to decorate memory cells with permissions. Although primarily
	meant as introduction to the different logical and syntactical fundamentals
	of \emph{Separation Logic} and \emph{Permission Accounting} a
	few properties can be shown in this paper like the \emph{Passivity} of commands on memory
	cells for which only read permission is given or the soundness of the
	\emph{Frame Rule} for sequential programs which use \emph{Permission
	Accounting}.
