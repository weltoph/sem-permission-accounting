\section{Conclusion}
%	\begin{itemize}
%		\item way to reason about heaps
%		\item way to resourcing
%		\item basis for tool assisted proofs
%	\end{itemize}
	In conclusion, \emph{Separation Logic} was introduced as a tool to reason
	about mutable data structures in the \emph{Heap}. Therefore the syntax and
	semantics of the logic were introduced. To work with concurrently executed
	programs the fundamentals of sharing memory are introduced. Furthermore,
	\emph{Permission Accounting} was defined to grant different permissions on
	memory cells and the \emph{Passivity} of programs on memory cells for which
	only read permission is given was proven. As mentioned the soundness and
	race-freedom of this approach of \emph{Permission Accounting} are proven in
	\cite{seplogproof}. This approach leads to interesting further ideas.
	It is possible to adapt \emph{Permission Accounting} for the use in the
	\emph{Store} \cite{storeperm}. Furthermore, the introduced abstraction of
	the models for \emph{Permission Accounting} can be used to develop other
	models which include other useful properties. In \cite{freshlook} for example
	the property of \emph{Splittability} is introduced which states that a
	permission can be infinitly splitted. As stated in \cite{PermAcc}
	it is desirable to expand the term \emph{ressources} which can be shared
	between concurrent programs to fit everything a program can manage(e. g.,
	the used space, the processing time,\dots). And also, that the specification
	of every program has to explicitly state which ressources are required. And
	of course reasoning about programs is always interesting for computer aided
	verification\footnote{smallfoot - a verification tool, based on
	\emph{Separation Logic}; see http://www0.cs.ucl.ac.uk/staff/p.ohearn/smallfoot/}
	\cite{freshlook}.
