\section{On Separation Logic}
	%Motivation:
	\emph{Separation Logic} is an extension of the \emph{Hoare Calculus}.
	It was developed to reason about programs which work with mutable data
	structures within the dynamically allocatable memory. In order to deal with
	dynamically allocatable memory a model later introduced as
	\emph{Heap}, will be used which represents the allocated memory as
	addressable cells. Also, a model for the variables in a program is
	introduced which will be used to keep track of the values of program
	variables in a central instance. This will be later on refered to as
	\emph{Store}. The following introduction of \emph{Hoare Calculus} is
	based on \cite{hoarecalc} and the introduction to \emph{Separation
	Logic} mainly on \cite{Primer}.

	\subsection{Preliminaries}
	Because the following section relies on operating with partial finite
	functions a few syntactical simplifications are used, namely:
	For two finite partial functions $f, f'$ the syntax $f \# f'$ is used
	to express disjunct domains of those functions(i.e.
	$\textit{dom}(f)\cap\textit{dom}(f') = \emptyset$). Also, introducing
	the union of finite partial functions as:
	\begin{mydef}[Union of Finite Partial Functions]
		For two finite partial functions $f_1, f_2 : D \rightharpoonup V$ the
		union of those (written as
		$f' = (f_1\bullet f_2) : D \rightharpoonup V$) is defined if $f_1 \# f_2$
		as
		$$
			f'(x) =
			\begin{cases}
				f_1(x)             &\text{if } x \in \textit{dom}(f_1)\\
				f_2(x)             &\text{if } x \in \textit{dom}(f_2)\\
				\textit{undefined} &\text{otherwise}\\
			\end{cases}
		$$
	\end{mydef}

	Finally, $f' = (f|i\mapsto j)$ defines a finite partial function
	which mirrors the mapping of $f$ except that $i$ maps to $j$, formally:
	$$(f|i\mapsto j) (n) =
		\begin{cases}
			j &\text{if } n = i\\
			f(n) &\text{if } n \in \textit{dom}(f)\setminus\{i\}\\
			\text{undefined} &\text{otherwise}\\
		\end{cases}
	$$

	\subsection{\emph{Hoare Calculus}}
	\emph{Hoare Calculus} was introduced in \cite{hoarecalc} as an approach to
	give assertions for a computer program. This approach focuses on an deductive
	reasoning about the program code which is written in a formal language. For
	simplicity usually a small language whit a limited amount of commands is
	used. Although these commands are sufficient to define all computable functions
	this approach can be applied to more complex languages in the
	same fashion. The used language lacks (for now) the ability to deal with
	pointers and allocations of dynamically managed memory. This will be added
	further on. The assertions for the program are expressed using mathematical
	logic, specifically first-order logic. Because the values of variables
	after a program's execution are often dependent on their input values, assertions
	can be given which are expected before a program
	executes. To refer to an assertion which states properties of the variables
	before the program's execution the term \emph{precondition} is used and for
	an assertion after the program's execution the term \emph{postcondition}.
	\emph{Hoare Calculus} can be used to check a \emph{postcondition} for a
	program by applying defined rules or axioms (which will be introduced
	later on) to find the necessary \emph{precondition}.

	To connect a \emph{precondition} $P$, a program $Q$ and a \emph{postcondition}
	$R$ the following syntax of Hoare-triple is used:
	$$ \{P\}Q\{R\} $$
	Note that $R$ refers to the moment after the execution of $Q$ but as one
	easily can imagine there exist programs which do not terminate because a loop
	is executed infinitly or fail to
	execute because of technical limitations like insufficient memory space or
	attempted access to memory which is denied by the operating system. In
	such a case the \emph{postcondition} $R$ is never reached, which makes every
	\emph{postcondition} valid. Thus, the following definition for correctness 
	takes into account that programs might not terminate by assuring the
	\emph{postcondition} only for terminating programs:
	\begin{mydef}[partial correctness]
		The Hoare-triple of a \emph{precondition} $P$, a program $Q$ and a
		\emph{postcondition} $R$
		$$ \{P\}Q\{R\} $$
		is called \emph{partially correct} if given
		$P$ is true before the execution of $Q$, $R$ is true after $Q$ terminates
		succesfully.
	\end{mydef}
	Termination implies that no extern instance like the
	operating system stopped the program, but the sequence of commands of the
	program ended. In order to reason about complex programs it is helpful if not
	necessary to define sound axioms and rules which can be applied to prove
	programs. Some rules and axioms are
	given below, but for a more comprehensive introduction see \cite{hoarecalc}:

	One of the basic commands of a programming language is the assignment
	of a value to a variable. This is expressed in the used formal language as
	$$x:=E$$
	where $E$ is any expression consisting of defined operators and variables.
	In order to keep this introduction short the set of operators is not defined
	but it can be assumed to consist of basic mathematical operators like
	$+,-,\cdot,\dots$
	An assertion which is true after an assignment $x:=E$ needs to be true before
	the assignment but where the expression $E$ replaces all occurences of $x$.
	Using a simplifying syntax $P[x/E]$ for an assertion $P$ where every occurance
	of $x$ is replaced by the expression $E$, this can be defined as the \emph{Axiom of Assignment}:
	\begin{mydef}[Axiom of Assignment]
		For an assertion $P$ the Hoare-triple
			$$\{P[x/E]\} x:=E \{P\}$$
		is partially correct.
	\end{mydef}

	Next, a rule is given which is used to reason about programs respecting
	their sequential execution. Normally a program
	consists of different commands which are executed one after another. Therefore
	the notation $Q_1;Q_2$ is used to express a composition of programs where
	the programpart $Q_1$ is executed and after its successful termination $Q_2$ is
	executed.
	Note that $Q_i$ can be a composition of programs itself and this way
	arbitrary complex programs can be considered by proving every command on its
	own. Now, if $Q_1$
	guarantees a certain \emph{postcondition} which is
	the \emph{precondition} for $Q_2$ then the \emph{postcondition} for $Q_2$ can
	be guaranteed for the composition $Q_1;Q_2$ if the \emph{precondition} of $Q_1$
	is met. More formally this can be expressed in the \emph{Rule of Composition}:
	\begin{mydef}[Rule of Composition]
		If $\{P\}Q_1\{T\}$ and $\{T\}Q_2\{R\}$ are partially correct so is
		   $\{P\}Q_1;Q_2\{R\}$.
	\end{mydef}

	To prove a sequence of statements it can be useful to give the assertions a
	certain form to meet a rule or axiom. But this change in form needs to be
	sound and therefore a rule which allows to use logical implications of assertions for
	proving correctness of Hoare-triples is introduced. To separate implications
	of assertions and implications in an assertion two syntactical types of
	logical implication are distinguished. In an assertion
	the notation $P\Rightarrow R$ defines a logical implication from two logical
	terms $P$ to $R$,
	but for assertions $P_1$, $P_2$ the notation $\{P_1\} \therefore \{P_2\}$ will
	be used to express a logical implication from one assertion to another. Now
	assume there is a Hoare-triple $\{P\}Q\{S\}$ which is partially correct. And
	$\{S\}\therefore\{R\}$ is true. Obviously $\{P\}Q\{R\}$ has to be
	partially correct as well, because with a \emph{precondition} $P$ a more strict
	\emph{postcondition} $S$ is given which logically implies $R$. In the same way
	a more strict \emph{precondition} which logically implies a sufficient
	\emph{precondition} can be used instead of the less strict but sufficient
	\emph{precondition}. This is called the \emph{Rule of Consequence} and can be
	formally expressed as:
	\begin{mydef}[Rule of Consequence]
		If $\{P\}Q\{R\}$ is partially correct and $\{R\}\therefore\{S\}$
		then $\{P\}Q\{S\}$ is partially correct.\\
		If $\{P\}Q\{R\}$ is partially correct and $\{S\}\therefore\{P\}$
		then $\{S\}Q\{R\}$ is partially correct.
	\end{mydef}


	To conclude this small introduction to \emph{Hoare Calculus} a last rule
	is introduced. The \emph{Rule of Iteration} is used to prove properties of
	loops. To begin with, the syntax of a loop for the used programming language
	is introduced. For a boolean expression $B$ and a programpart $C$ the syntax
	\begin{center} while $B$ do $C$ done \end{center} is used to express the
	execution of $C$ while $B$ is evaluated to true. In order to give assertions
	for loops loop-invariants are used. A loop-invariant is an assertion which
	is true before the loop and can be guaranteed after every execution of $C$.
	To give a simpel example the assertion $\{\textit{true}\}$ is always for every
	loop an invariant, but of course one with little value regarding assertions
	of a program's properties. But because a loop-invariant is true after every
	execution of the loop and before the loop it is especially true after the
	loop. And after the loop it is evident that $B$ can not evaluate to true.
	Therefore after a loop the invariant as well as the negated condition can
	be asserted for the program state, formally defined:
	\begin{mydef}[Rule of Iteration]
		For a boolean expression $B$, an assertion $P$ and a program $C$: If
		$\{P \land B\}C\{P\}$ is partially correct, so is
		$\{P\}\text{while } B \text{ do } C \text{ done}\{\lnot B \land P\}$.
	\end{mydef}


	\subsection{\emph{Store} and \emph{Heap}}
	%kurze Einführung was Heap und Store sind:
	As mentioned above \emph{Hoare Calculus} does not deal explicitly with
	dynamically allocatable memory and mutable structures within the memory.
	But to reason about programs which use dynamic allocation of memory it is
	useful to separate the memory into two parts, the \emph{Store} and the
	\emph{Heap}. The \emph{Store} is used in \emph{Hoare Calculus} because by
	using program variables within assertions it is implicity assumed that those
	variables are somehow mapped to a value.
	This finds formalisation in the \emph{Store} which maps the
	used variables in a program to integer values. The used variables of
	a program are a subset of all possible variables, the set \emph{Vars}.
	And the set of integer values is refered to as \emph{Integers}, note
	that this set might depend on the machine the program is executed on.
	
	\begin{mydef}[\emph{Store}]
		A \emph{Store} is a finite partial function
		$s:\textit{Vars}\rightharpoonup \textit{Integers}$
	\end{mydef}
	\begin{myexp}[\emph{Store}]
	A possible representation of a \emph{Store} $s$.\\
	\begin{center}
	\begin{tikzpicture}
		\matrix (store) [table, text width=7mm, name=table]
			{ x & y & z \\
			 -7 & 3 & 7 \\ };
	\end{tikzpicture}
	\end{center}
	Here the variable
	$x$ holds the value $-7$, $y$ the value $3$ and $z$ $7$, thus:
	$s(x) = -7$, $s(y) = 3$, $s(z) = 7$.
	Furthermore $\text{dom}(s) = \{x,y,z\}$.
	\end{myexp}

	The \emph{Heap} is a formalisation of addressable memory which can be
	dynamically allocated by a program. It manages memory cells which contain
	integer values and are uniquely identified by a natural number as their
	address.
	\begin{mydef}[\emph{Heap}]
		A \emph{Heap} is a finite partial function
		$h:\mathbb{N} \rightharpoonup \textit{Integers}$
	\end{mydef}
	\begin{myexp}[\emph{Heap}]
	A possible state for a \emph{Heap} $h$.\\
	\begin{center}
	\begin{tikzpicture}
		\matrix (store) [table, text width=7mm, name=table]
			{ 3 &  4 &  5 &  6 \\
			  7 & 12 & -6 & -1 \\ };
	\end{tikzpicture}
	\end{center}
	\label{fig:heap}
	 Here the cell with address
	$3$ holds the value $7$, the cell with address $4$ the value 12,
	the cell with address $5$ the value -6 and the cell with address $6$ the
	value -1,
	thus: $h(3) = 7$, $h(4) = 12$, $h(5) = -6$, $h(6)=-1$.
	Furthermore $\text{dom}(h) = \{3,4,5,6\}$.
	\end{myexp}

	%semantik, wie man store Variablen interpretiert:
	Because the mapping of the program's variables is now formalised by the
	\emph{Store} an expression $E$ which is for example used in the
	\emph{Axiom of Assignment} in \emph{Hoare Calculus} needs interpretation.
	The \emph{Store} gives the syntactical objects of variables their values. Therefore
	the syntax $\llbracket E \rrbracket s$ will be used to refer to the expression
	$E$ where all variables are interpreted by the \emph{Store} $s$. Referring to
	the example of the \emph{Store} above the expression $x + y$ is evaluated to
	$\llbracket x + y \rrbracket s \equiv s(x) + s(y) = -7 + 3 = 4$. The same
	applies to boolean expressions $B$ which are evaluated over a \emph{Store}
	$s$ written as $\llbracket B \rrbracket s$. For example using the \emph{Store}
	$s$ from the example above $x + 7 > z$ is evaluated to
	${\llbracket x + 7 > z \rrbracket s} {\;\equiv s(x) + 7 > s(z)}
	 {\;\equiv -7 + 7 > 7} {\;\equiv  0 > 7} {\;\equiv \textit{false}}$.

	\subsection{Logical Operators}
	In order to use the model of \emph{Stores} and \emph{Heaps} the used logic
	has to be able to state properties of those.
	To this aim the logic is enriched with some expressions for which an intuition
	will be given initially and which will be then defined in terms of
	\emph{Heaps} and \emph{Stores} later on.

	Firstly, the "$\mapsto$ operator" is used to connect the address of one cell
	with its content. For example $5\mapsto 2$ states that the cell which is
	addressed by $5$ holds the value $2$ or more generally, $E\mapsto F$ states
	that the value of the cell which is addressed by the value of the expression
	$E$ is the value of the expression $F$.

	Secondly, the $\ast$-operator is used to connect two assertions which are
	given for two disjunctive parts of the current state of the \emph{Heap}.
	That means $P\ast Q$ for two assertions $P,Q$
	expresses that the allocated memory cells can be split into two independent
	parts where one satisfies $P$ and the other satisfies $Q$.

	Thirdly, the expression \textit{emp} is used to express that there are no cells
	in the \emph{Heap} allocated yet.

	Fourthly, the meaning of the $\forall$-operator slightly changes to its common
	interpretation because the
	\emph{Store} is used to map variables to values. Because the statement
	$\forall x.P$ for an assertion $P$ needs to be true for
	a \emph{Heap} and a \emph{Store} for every possible value $x$ can hold. But
	$x$ is evaluated through the \emph{Store} which means that the definition of
	$\forall$ is adjusted to take that into account. This applies in the same way
	to the $\exists$-operator.

	With this intuition in mind the definitions of the operator for \emph{Stores}
	and \emph{Heaps} follow naturally. For those
	assume that $s$ is a \emph{Store}, $h,h_1,h_2$ are \emph{Heaps}, $B$ is a
	boolean expression and $P$ is an assertion, and the syntax $s, h \models P$
	means that the assertion $P$ holds for the current state of the \emph{Store}
	$s$ and the \emph{Heap} $h$.
	\begin{mydef}[Logical Assertions]
		The semantic definitions for the \emph{logical assertions} are:\\
		\begin{center}
		\begin{tabular}{lll}
			$s, h \models E\mapsto F$ &iff &$\llbracket E\rrbracket s  = \textit{dom}(h) \text{ and }
			                               h(\llbracket E\rrbracket s) = \llbracket F\rrbracket s$ \\
			&&\\
			$s, h \models P\ast Q$ &iff &$\exists h_0,h_1.\;h_0\#h_1,\;h_0\bullet h_1=h,\;
			s,h_0 \models P \text{ and } s,h_1\models Q$\\
			&&\\
			$s, h \models\textit{emp}$ &iff  &$\textit{dom}(h) = \emptyset$\\
			&&\\
			$s, h \models B$ &iff &$\llbracket B\rrbracket s \equiv \textit{true}$\\
			&&\\
			$s, h \models \forall x.P$ &iff &$\forall v \in \textit{Integers}.(s|x\mapsto v),h \models P$\\
			&&\\
			$s, h \models \exists x.P$ &iff &$\exists v \in \textit{Integers}.(s|x\mapsto v),h \models P$\\
		\end{tabular}
		\end{center}
	\end{mydef}
	Note that for $E\mapsto F$ the \emph{Heap} consists of only one cell, but
	with $\ast$ more complex \emph{Heaps} can be described. For example for the
	\emph{Heap} $h$ and the \emph{Store} $s$ from the
	examples above applies $s, h \models y\mapsto 7\ast 4\mapsto 12
	\ast 5\mapsto x+1\ast 6\mapsto -1$.

	\subsection{Program Syntax}
	As mentioned in the introduction to \emph{Hoare Calculus} the formal language
	used to define programs lacks a syntax of accessing memory cells in the
	\emph{Heap}. In order to introduce such a syntax and connect it to the
	corresponding logical assertions further axioms are introduced below.

	First of all, there is no explicit way to state that a variable is used as
	reference to a memory cell. Rather every
	natural number can be interpreted as pointer. Therefore the syntax $[n]$ for
	$n\in\mathbb{N}$ is used as the content of the cell with address $n$.
	For example if for the current \emph{Heap} $h$ the following holds: $3\in\textit{dom}(h)$
	and $h(3) = 5$ then in the program the condition statement $([3] = 5)$ would evaluate to
	\textit{true}. For the definition of the new axioms one syntactical simplification
	is introduced as $E \mapsto -$ which is an abbrevation for
	$\exists y.E\mapsto y$. For the following definitions $E,F$ are integer
	expressions(expressions that are evaluated to an integer value by the
	programming language), $m$ is an integer and $x$ is a variable currently defined in
	the \emph{Store} $s$.
	\newcounter{AxiomIndex}
	\setcounter{AxiomIndex}{1}
	\begin{mydef}[Axioms of Referenced Access]
		The Hoare-triples\\
		\begin{center}
		\begin{tabular}{llll}
			\arabic{AxiomIndex}\stepcounter{AxiomIndex}&
			$\{ E \mapsto - \}$ &$[E] := F$ &$\{ E \mapsto F \}$\\
			&&&\\
			\arabic{AxiomIndex}\stepcounter{AxiomIndex}&
			$\{E \mapsto n \land \llbracket x\rrbracket s = m\}$ &
				$x := [E]$ &$\{\llbracket x \rrbracket s=n \land E[m/x]\mapsto n\}$\\
			&&&\\
			\arabic{AxiomIndex}\stepcounter{AxiomIndex}&
			$\{E \mapsto - \}$ &$\textit{dispose}(E)$ &$\{\textit{emp}\}$\\
			&&&\\
			\arabic{AxiomIndex}\stepcounter{AxiomIndex}&
			$\{\textit{emp}\}$ &$x := \textit{new}()$ &$\{x\mapsto -\}$\\
			&&\\
		\end{tabular}
		\end{center}
		are partially correct.
	\end{mydef}
	Note that new() returns a memory address which is not yet in the domain
	of the \emph{Heap} and in the second axiom the assignment
	of a value to $x$ can change the value of $E$. This is the reason
	for using the logical variable $m$ which persists through the assignment
	to replace occurences of $x$ in $E$. The definition of \textit{dispose} is
	taken from \cite{PermAcc}.

	In order to conclude the section about \emph{Separation Logic} the
	\emph{Frame Rule} is introduced. The \emph{Frame Rule} is used to reason
	locally about programs. This means for any subprogram it is only necessary
	to give assertions about the heap-cells which are actually used and modified
	in the subprogram. Therefore any other cell in the \emph{Heap} does not have
	effect on the subprogram. The \emph{Frame Rule} is presented as interference
	rule:
	\begin{mydef}[Frame Rule]
		For a program $C$ and assertions $P,Q,R$ holds:
		\begin{prooftree}
			\AxiomC{$\{P\}C\{Q\}$}
			\RightLabel{$\textit{modifies}(C)\cap\textit{free}(R) = \emptyset$}
			\UnaryInfC{$\{P\ast R\}C\{Q\ast R\}$}
		\end{prooftree}
	where $\textit{modifies}(C)$ is the set of variables modified by $C$ and
	$\textit{free}(R)$ the set of free variables in $R$.
	\end{mydef}
	These are the used rules and axioms for \emph{Separation Logic} in this paper.
	But as stated in \cite{freshlook} it is possible to work with other approaches
	to model the memory and mutable data structures therefore the presented
	axioms and rules are only a representative set.
	Following one small example will be examined:
	\begin{myexp}[Highest Value of Array]
		This example is a procedure which determines the highest value
		of an array. Because there is no defined syntax for an array it
		is expected to be a consecutive block of memory cells. The procedure
		takes a pointer to the first cell of the array and the length of the
		array. Firstly, a possible implementation is given, where \emph{skip}
		is the command that does nothing:
		\begin{lstlisting}[mathescape]
procedure maxValue($l$,$n$) {
	maxV := [l];
	m := 1;
	while(m < n) {
		if(maxV < [l + m]) {
			maxV := [l + m];
		} else {
			skip;
		}
		m := m + 1;
	}
}
		\end{lstlisting}
		Now in place reasoning is used to prove that \emph{maxV} will hold
		the maximal value of the array at the end of the procedure. For the
		programming lines comments are given which states the axiom or rule used to
		modify the assertions.
		\begin{lstlisting}[mathescape,tabsize=2,escapeinside={-<}{>-}]
-<\textcolor{green}{$\{l\mapsto v_1\ast l+1\mapsto v_2\ast\cdots\ast l+n-1\mapsto v_n\land\llbracket n\rrbracket s > 0\}$}>-
// the array needs to have at least one element
// and to be a consecutive block of memory cells
procedure maxValue($l$,$n$) {
-<\textcolor{green}{$\{l\mapsto v_1\ast l+1\mapsto v_2\ast\cdots\ast l+n-1\mapsto v_n\land\llbracket n\rrbracket s > 0\}$}>-
-<\textcolor{green}{$\therefore\{l\mapsto v_1\ast l+1\mapsto v_2\ast\cdots\ast l+n-1\mapsto v_n\land\llbracket n\rrbracket s>0 \land$}>-
-<\textcolor{green}{$\llbracket maxV \rrbracket s = \llbracket maxV \rrbracket s \land \llbracket m \rrbracket s = \llbracket m \rrbracket s\}$}>-
	maxV := [l]; // Axioms of Referenced Access
	m := 1; // Axiom of Assignment
	-<\textcolor{green}{$\{l\mapsto v_1\ast l+1\mapsto v_2\ast\cdots\ast l+n-1\mapsto v_n\land\llbracket n\rrbracket s>0 \land$}>-
	-<\textcolor{green}{$\llbracket maxV \rrbracket s = v_1 \land \llbracket m \rrbracket s = 1\}$}>-
	// loop invariant is: $m \leq n \land maxV = \max_{1\leq j\leq m}v_j$
	-<\textcolor{green}{$\therefore\{l\mapsto v_1\ast l+1\mapsto v_2\ast\cdots\ast l+n-1\mapsto v_n\land\llbracket n\rrbracket s>0 \land$}>-
	-<\textcolor{green}{$m \leq n \land \llbracket maxV\rrbracket s = \max_{1\leq j\leq m}v_j\}$}>-
	while(m < n) {
	-<\textcolor{green}{$\{l\mapsto v_1\ast l+1\mapsto v_2\ast\cdots\ast l+n-1\mapsto v_n\land\llbracket n\rrbracket s>0 \land$}>-
	-<\textcolor{green}{$m \leq n \land \llbracket maxV\rrbracket s = \max_{1\leq j\leq m}v_j\land \llbracket m \rrbracket s < \llbracket n \rrbracket s\}$}>-
		if(maxV < [l + m]) { // if-Rule
			-<\textcolor{green}{$\{l\mapsto v_1\ast l+1\mapsto v_2\ast\cdots\ast l+n-1\mapsto v_n\land\llbracket n\rrbracket s>0 \land$}>-
			-<\textcolor{green}{$\llbracket m\rrbracket s \leq \llbracket n\rrbracket s \land \llbracket maxV\rrbracket s = \max_{1\leq j\leq m}v_j$}>-
			-<\textcolor{green}{$\land \llbracket m \rrbracket s < \llbracket n \rrbracket s \land \llbracket maxV \rrbracket s < v_{m+1}\}$}>-
			-<\textcolor{green}{$\therefore\{l\mapsto v_1\ast l+1\mapsto v_2\ast\cdots\ast l+n-1\mapsto v_n\land\llbracket n\rrbracket s>0 \land$}>-
			-<\textcolor{green}{$\llbracket m\rrbracket s \leq \llbracket n\rrbracket s \land v_{m+1} = \max_{1\leq j\leq m + 1}v_j$}>-
			-<\textcolor{green}{$\land \llbracket m \rrbracket s < \llbracket n \rrbracket s \}$}>-
			maxV := [l + m]; // Axioms of Referenced Access
			-<\textcolor{green}{$\{l\mapsto v_1\ast l+1\mapsto v_2\ast\cdots\ast l+n-1\mapsto v_n\land\llbracket n\rrbracket s>0 \land$}>-
			-<\textcolor{green}{$\llbracket m\rrbracket s\leq\llbracket n\rrbracket s\land \llbracket maxV\rrbracket s = \max_{1\leq j\leq m + 1}v_j$}>-
			-<\textcolor{green}{$\land\llbracket m\rrbracket s<\llbracket n\rrbracket s\}$}>-
		} else {
			skip;
			-<\textcolor{green}{$\{l\mapsto v_1\ast l+1\mapsto v_2\ast\cdots\ast l+n-1\mapsto v_n\land\llbracket n\rrbracket s>0 \land$}>-
			-<\textcolor{green}{$\llbracket m\rrbracket s \leq \llbracket n\rrbracket s \land \llbracket maxV\rrbracket s = \max_{1\leq j\leq m}v_j$}>-
			-<\textcolor{green}{$\land \llbracket m \rrbracket s < \llbracket n \rrbracket s \land \llbracket maxV \rrbracket s \geq v_{m+1}\}$}>-
			-<\textcolor{green}{$\therefore\{l\mapsto v_1\ast l+1\mapsto v_2\ast\cdots\ast l+n-1\mapsto v_n\land\llbracket n\rrbracket s>0 \land$}>-
			-<\textcolor{green}{$\llbracket m\rrbracket s \leq \llbracket n\rrbracket s \land \llbracket maxV\rrbracket s = \max_{1\leq j\leq m + 1}v_j$}>-
			-<\textcolor{green}{$\land \llbracket m \rrbracket s < \llbracket n \rrbracket s \}$}>-
		}
		-<\textcolor{green}{$\{l\mapsto v_1\ast l+1\mapsto v_2\ast\cdots\ast l+n-1\mapsto v_n\land\llbracket n\rrbracket s>0 \land$}>-
		-<\textcolor{green}{$\llbracket m\rrbracket s \leq \llbracket n\rrbracket s \land \llbracket maxV\rrbracket s = \max_{1\leq j\leq m + 1}v_j$}>-
		-<\textcolor{green}{$\land \llbracket m \rrbracket s < \llbracket n \rrbracket s \}$}>-
		-<\textcolor{green}{$\therefore\{l\mapsto v_1\ast l+1\mapsto v_2\ast\cdots\ast l+n-1\mapsto v_n\land\llbracket n\rrbracket s>0 \land$}>-
		-<\textcolor{green}{$\llbracket m+1\rrbracket s \leq \llbracket n\rrbracket s \land \llbracket maxV\rrbracket s = \max_{1\leq j\leq m + 1}v_j\}$}>-
		m := m + 1; // Axiom of Assignment
		-<\textcolor{green}{$\{l\mapsto v_1\ast l+1\mapsto v_2\ast\cdots\ast l+n-1\mapsto v_n\land\llbracket n\rrbracket s>0 \land$}>-
		-<\textcolor{green}{$\llbracket m\rrbracket s \leq \llbracket n\rrbracket s \land \llbracket maxV\rrbracket s = \max_{1\leq j\leq m}v_j\}$}>-
	}
	-<\textcolor{green}{$\{l\mapsto v_1\ast l+1\mapsto v_2\ast\cdots\ast l+n-1\mapsto v_n\land\llbracket n\rrbracket s>0 \land$}>-
	-<\textcolor{green}{$\llbracket m\rrbracket s \leq \llbracket n\rrbracket s \land \llbracket maxV\rrbracket s = \max_{1\leq j\leq m}v_j$}>-
	-<\textcolor{green}{$\land \llbracket m\rrbracket s\geq \llbracket n\rrbracket s\}$}>-
	-<\textcolor{green}{$\therefore\{l\mapsto v_1\ast l+1\mapsto v_2\ast\cdots\ast l+n-1\mapsto v_n\land\llbracket n\rrbracket s>0 \land$}>-
	-<\textcolor{green}{$\llbracket m\rrbracket s = \llbracket n\rrbracket s \land \llbracket maxV\rrbracket s = \max_{1\leq j\leq m}v_j\}$}>-
	-<\textcolor{green}{$\therefore\{l\mapsto v_1\ast l+1\mapsto v_2\ast\cdots\ast l+n-1\mapsto v_n\land\llbracket n\rrbracket s>0 \land$}>-
	-<\textcolor{green}{$\land \llbracket maxV\rrbracket s = \max_{1\leq j\leq n}v_j\}$}>-
}
		\end{lstlisting}
	As this example shows it is a daunting task to reason about programs by hand.
	But the assertions above prove that the variable \emph{maxV}
	holds the highest value of the array, just as requested.
	\end{myexp}
